%\part*{Introduction}
\chapter*{Introduction générale}
\markboth{Introduction générale}D
Dans le domaine du BTP, ici en Algérie, il existe quelques moyens tels que les
réseaux sociaux et des sites web tels que Ouedkniss, qui permet aux entreprises
et aux individus de proposer leurs services et marchandises d’une manière
anarchique et qui n’assure surtout pas l'évaluation des services et
marchandises.
\\

\par Soit pour les grands projets qui nécessitent des entreprises pour les
réaliser, soit pour les petits projets qui nécessitent un seul travailleur ou
plusieurs, le client rencontre toujours des problème lors de la recherche de la
bonne personne à recruter, car il existe pas une platform ou un organisme qui
permet l’évaluation et recommandation de tell services. Un autre problème qu’on
rencontre souvent aussi ici en Algérie, c’est le problème de prix, si vous
cherchez les prix de n'importe quelle matière essentielle de batiment en ligne
(en utilisant google par exemple), vous ne trouverez pas grand chose. Pour une
personne qui ne connaît rien du BTP et des prix des matières du BTP, il peut se
tromper facilement.\\

\par Notre projet est de concevoir une application web qui permet de résoudre
cette problématique d’un côté et d'offrir des services de qualité soit pour les
travailleurs ou les clients et permettre aussi d'éliminer l'ambiguïté sur le
marché du BTP  ici en Algérie d’un autre côté.\\
\newpage
\vspace{0.5cm}
\par Notre mémoire se présente en un document structuré en quatre chapitres
comme suit : 

\begin{itemize}[label=\textbullet]
\item \textbf{Chapitre 1 " Définition et généralité sur le système
d'information" : } ------ \\

\item \textbf{Chapitre 2 " ******* " : } ------\\

\item \textbf{Chapitre 3 " Conception " :} ---------\\  

\item \textbf{Chapitre 4 " Réalisation" : }-------
\end{itemize}

\par Enfin, nous clôturons ce mémoire par une conclusion dans laquelle nous
résumons notre mémoire et nous exposons quelques perspectives
futures.






















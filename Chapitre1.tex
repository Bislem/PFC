\chapter{Définition et généralité sur le système d'information }
\section{Introduction }

\par Dans ce chapitre nous allons decouvrire les system d'information et leur
differentes aspects qui nous permetrea de construire notre system d'information
par rapport a notre application


\section{Définition d'un système d'information }
\par Les systèmes d'information sont des composants interdépendants qui
travaillent ensemble pour collecter, traiter, stocker et diffuser des
informations pour soutenir la prise de décision, la coordination, le contrôle,
l'analyse et la visualisation dans une organisation.
\cite{ref1}.

\section{Composants des systèmes d'information }
\par chaque système d'information est constitués de cinq composants clés: le
matériel, les logiciels, les télécommunications, les personnes et les données.
Le matériel fait référence aux éléments physiques du système d'information ; le
logiciel est la programmation qui contrôle le système d'information ; la
télécommunication transmet des informations à travers le système ; les humains
gèrent et interagissent avec le système d'information ; et les données sont des
informations stockées dans et traitées par le système.

\subsection{Matérial }
\par Le composant matériel d'un système d'information comprend les éléments
physiques du système, qui peuvent être touchés et senti. Ces mécanismes,
équipements et câblages permettent à des systèmes tels que les ordinateurs, les
smartphones et les tablettes de fonctionner.\\
Les périphériques d'entrée et de sortie tels que les souris, claviers,
imprimantes et moniteurs sont des éléments technologiques essentiels qui
permettent aux humains d'interagir avec les ordinateurs et d'autres systèmes
d'information. et d’autres composants tels que les microprocesseurs, les
disques durs, les blocs d'alimentation, permettent également aux ordinateurs de
stocker et de traiter des données.

\subsection{Logiciel }
\par Ce sont les programmes utilisés pour organiser, traiter et analyser les
données.

\subsection{Télécommunications }
\par Différents éléments doivent être connectés les uns aux autres, permettent
et facilitent la circulation de l’information dans l’organisation

\subsection{Données }
\par les données sont des faits et des chiffres bruts qui ne sont pas organisés
et qui sont ensuite traités pour générer des informations.

\subsection{Ressources humaines }
\par Le personnel est un élément des plus important à tout système
d’information, elle regroupe tous les utilisateurs que ce soit les utilisateurs
finaux ou les gestionnaires de système d’information, l’un sollicite ces
services et l’autre assure son bon fonctionnement.

\section{Fonctions et fonctionnement des systèmes d’informations }
\par il existe quatre fonctions principales d'un systemes d'information:
collecter, stocker, traiter et diffuser l’information.

\subsection{Collecter }
\par Pour qu'un système fonctionne, il doit être alimenté. Les informations
collectées proviennent de sources internes ou externes. 

\subsection{Stocker }
\par Une fois l’information collectée, il faut garantir un stockage durable et
fiable.et il peut être sous différente forme tel que les base de données ou
fichiers.

\subsection{Traiter }
\par Pour être exploitable, l’information subit des traitements. et cela
consiste à produire de nouvelles informations à partir d’informations
existantes grâce à des programmes informatiques ou des opérations
manuelles(c’est de moins en moins souvent le cas).
Le traitement de l’information peut prendre différentes formes, comme consulter
juste l’information de base sans faire aucune modification, organiser
l’information en fonction des critères spécifiques, modifier l’information ou
supprimer les informations non pertinentes.

\subsection{Diffuser l'information }
\par Pour qu’une information aie de la valeur, elle doit être transmis à la
bonne personne, organisme dans les meilleurs délais. 

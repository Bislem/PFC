\documentclass[12pt, letterpaper]{article}
\usepackage[utf8]{inputenc}
\usepackage[T1]{fontenc}
\usepackage[french]{babel}
\bibliography{bibliographie}



\begin{document}
\chapter{Définition et généralité sur le système d'information}
\begin{center}
    \huge{Chapitre 1: Définition et généralité sur le système d'information}
\end{center}

\section{Introduction}
De nombreuses organisations travaillent avec de grandes quantités de
données . Les données sont des valeurs ou des faits de base et sont
organisées dans une base de données . Beaucoup de gens pensent que
les données sont synonymes d' informations ; cependant, l'information
consiste en fait en des données qui ont été organisées pour aider à
répondre à des questions et à résoudre des problèmes. Un système
d'information est défini comme le logiciel qui permet d'organiser et
d'analyser les données. Ainsi, le but d'un système d'information est 
de transformer des données brutes en informations utiles pouvant être
utilisées pour la prise de décision dans une organisation.

\section{Définition d'un système d'information}
Les systèmes d'information sont des composants interdépendants qui
travaillent ensemble pour collecter, traiter, stocker et diffuser 
es informations pour soutenir la prise de décision, la coordination,
le contrôle, l'analyse et la visualisation dans une organisation.
\cite{Prog}

\section{Composants des systèmes d'information}
chaque système d'information est constitués de cinq composants clés: 
le matériel, les logiciels, les télécommunications, les personnes et
les données. Le matériel fait référence aux éléments physiques du 
système d'information ; le logiciel est la programmation qui contrôle 
le système d'information ; la télécommunication transmet des 
informations à travers le système ; les humains gèrent et interagissent
avec le système d'information ; et les données sont des informations 
stockées dans et traitées par le système.

    \subsection{Matérial}
     Le composant matériel d'un système d'information comprend les 
     éléments physiques du système, qui peuvent être touchés et 
     senti. Ces mécanismes, équipements et câblages permettent à 
     des systèmes tels que les ordinateurs, les smartphones et les 
     tablettes de fonctionner.\\
     Les périphériques d'entrée et de sortie tels que les souris, 
     claviers, imprimantes et moniteurs sont des éléments 
     technologiques essentiels qui permettent aux humains d'interagir 
     avec les ordinateurs et d'autres systèmes d'information. et 
     d’autres composants tels que les microprocesseurs, les disques 
     durs, les blocs d'alimentation, permettent également aux 
     ordinateurs de stocker et de traiter des données.
     
     \subsection{Logiciel}
     Ce sont les programmes utilisés pour organiser, traiter et 
     analyser les données.
      
     \subsection{Télécommunications}
     Différents éléments doivent être connectés les uns aux autres, 
     permettent et facilitent la circulation de l’information dans 
     l’organisation
     
     \subsection{Données}
     les données sont des faits et des chiffres bruts qui ne sont pas 
     organisés et qui sont ensuite traités pour générer des informations.
     
     \subsection{Ressources humaines}
     
     
\section{Les fonctions d’un système d’information}
il existe quatre fonctions principales d'un systemes d'information: 
collecter, stocker, traiter et diffuser l’information.


\end{document}
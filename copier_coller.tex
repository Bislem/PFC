\par Qu'est ce que le Référentiel des Données pour une entreprise ?
Il s'agit des informations qui constituent la base du Système d'Information de l'entreprise : données Clients, Produits, Employés, Fournisseurs, Structure de l'entreprise, ...
Ces données sont gérées dans les différents logiciels de gestion de l'entreprise : CRM, Comptabilité, Paie, Gestion commerciale, Gestion de production, Gestion d'entrepôts,
\par Nomenclature
Table de valeurs codifiées d’une donnée. Il s’agit d’une donnée de référence de type
paramètre.
Exemple : la donnée Activité principale de l’entreprise (ou code APE) est exprimée selon la Nomenclature des activités françaises (NAF).
\par
Un référentiel de données est un « répertoire » clairement identifié (qui « fait
référence »), qui stocke et permet la gestion et l’utilisation de données dans plusieurs
traitements
\par dentifiant
Une donnée spécifique qui repère de manière unique un objet ou une information
dans un système.
→ Un identifiant sert à repérer aussi les données de référence.
Exemple : le numéro SIRET identifie les établissements d’une entreprise.
\par
Référentiel
Un référentiel s’apparente à un ensemble d’éléments dans lequel l’entreprise documente et agrège des règles de fonctionnement, techniques ou fonctionnelles.

\par Information
Données agrégées en vue d’une utilisation par l’homme (par exemple, le résultat
d’une requête décisionnelle qui somme des données individuelles est une information). On parle aussi d’élément de connaissance susceptible d’être représenté à l’aide
de conventions pour être conservé, traité ou communiqué (image, texte, donnée
structurée)
\par
l'étude de notre thème se déroule au niveau du groupe CEVITAL de Bejaia, en particulier au sein du départements de la transformation dégitale dans le but de la mise en place d'un système de gestion, centralisation et uniformisation des données.






\vspace{0.5cm}
\par Cependant, plus nous partageons l'information, plus nous réalisons que des années d'utilisation des outils informatiques et des applications d'entreprise entre les différents secteurs d'activité a conduit à des îlots de cohérence de donnée.
\vspace{0.5cm}
\par Dans ce contexte, et dans le cadre de notre projet de fin de cycle Master, nous avons entamé un stage au niveau de l'entreprise Cevital, qui a duré uniquement trois semaines malheureusement à cause du programme chargé de l'entreprise. Ce thème a été proposé  par le département de la transformation digitale. Malgré la durée du stage qui est petite, il nous a été très utile, pour bien étudier les besoins de l'entreprise.


